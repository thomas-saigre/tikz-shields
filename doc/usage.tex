%!TeX root=../tikz-shield-doc.tex
\section{Basic usage}

A badge can be included using the following command, in which the first argument specifies the left text (\textsf{Any text} in the example above), and the second argument is the text on the right (\textsf{you like}).

\begin{tcblisting}{title={\tikzshield}, lower separated=false}
\drawBadge{Any text}{you like}
\end{tcblisting}

The color of the right part can be customized, the default color being a rectangle of color \texttt{default-blue} \textcolor{default-blue}{\rule{1em}{1em}}.


\begin{tcblisting}{title={Change color}, lower separated=false}
\drawBadge[colorRight=red]{Any text}{you like}
\drawBadge[colorLeft=white]{Any text}{you like}
\drawBadge[colorLeft=pink, colorRight=black]{Any text}{you like}
\end{tcblisting}



\section{Optional arguments}

Some optional arguments can be passed to the macro \texttt{\textbackslash{}drawBadge} (as well as the other ones described in this document):

\begin{center}
    \begin{tblr}{
        colspec = {Q[c, cmd=\texttt]lc},
        row{1} = {cmd =  \normalfont, bg = default-blue!30},
        vlines, hlines,
    }
    Name & Description & Default value \\
    colorLeft & Color of the left part of the badge & \texttt{default-left} \textcolor{default-left}{\rule{1em}{1em}} \\
    colorRight & Color of the right part of the badge & \texttt{default-blue} \textcolor{default-blue}{\rule{1em}{1em}} \\
    logo & Logo displayed before the text &

    \end{tblr}
\end{center}




\section{Special badges}


Some presets are available :

\begin{tcblisting}{title={Github badge}, lower separated=false}
\githubBadge{thomas-saigre/tikz-shield}
\end{tcblisting}

\begin{tcblisting}{title={Gitlab badge}, lower separated=false}
\gitlabBadge[colorRight=orange]{inkscape/inkscape/}
\end{tcblisting}

\def\checkmark{\tikz\fill[scale=0.4](0,.35) -- (.25,0) -- (1,.7) -- (.25,.15) -- cycle;}

\begin{tcblisting}{title={Custom forge badge}, lower separated=false}
\forgeBadge[colorLeft=black, logo=\faCode]{code.videolan.org/videolan/vlc}
\forgeBadge[colorLeft=green, logo=\checkmark]{code.videolan.org/videolan/vlc}
\end{tcblisting}

In all cases, the generated badge is clickable and redirect to the repository provided.
For \texttt{\\forgeBadge}, the icon can be customized, as well as the color.
Note that anything can be used as a logo, even a text or a personalized command that you would have created (in previous example
\mintinline[breaklines]{tex}{\def\checkmark{\tikz\fill[scale=0.4](0,.35) -- (.25,0) -- (1,.7) -- (.25,.15) -- cycle;}}).
