%!TeX root=../tikz-shield-doc.tex
\section{Basic usage}

A badge can be included using the following command, in which the first argument specifies the left text (\textsf{Any text} in the example above), and the second argument is the text on the right (\textsf{you like}).

\begin{tcblisting}{title={\tikzshield}, lower separated=false}
\drawBadge{Any text}{you like}
\end{tcblisting}

The color of the right part can be customized, the default color being a rectangle of color \texttt{default-blue} \textcolor{default-blue}{\rule{1em}{1em}}.


\begin{tcblisting}{title={Change color}, lower separated=false}
\drawBadge[red]{Any text}{you like}
\drawBadge[white]{Any text}{you like}
\end{tcblisting}




\section{Special badges}


Some presets are available :

\begin{tcblisting}{title={Github badge}, lower separated=false}
\githubBadge{thomas-saigre/tikz-shield}
\end{tcblisting}

\begin{tcblisting}{title={Gitlab badge}, lower separated=false}
\gitlabBadge[orange]{thomas-saigre/tikz-shield}
\end{tcblisting}

\begin{tcblisting}{title={Custom forge badge}, lower separated=false}
\forgeBadge[black][\faCode]{code.videolan.org/videolan/vlc}
\end{tcblisting}

In all cases, the generated badge is clickable and redirect to the repository provided.
For \texttt{\\forgeBadge}, the icon can be customized, as well as the color.