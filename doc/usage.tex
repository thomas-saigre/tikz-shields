%!TeX root=../tikz-shields-doc.tex
\section{Basic usage}

A badge can be included using the command \texttt{\textbackslash drawBadge}, in which the first argument specifies the left text (\textsf{Any text} in the example below), and the second argument is the text on the right (\textsf{you like}).

\begin{tcblisting}{title={\tikzshields}, lower separated=false}
\drawBadge{Any text}{you like}
\end{tcblisting}


\section{Optional arguments}

Some optional arguments can be passed to the macro \texttt{\textbackslash{}drawBadge} (as well as the other ones described in this document):

\begin{center}
    \begin{tblr}{
        colspec = {Q[c,m, cmd=\texttt,0.18\textwidth]X[0.5\textwidth,m]X[0.17\textwidth,c,m]X[0.04\textwidth,c,m]},
        row{1} = {cmd=\textbf, bg=default-blue, fg=white},
        vlines, hlines,
        width = \textwidth,
    }
    Name & Description & Default value & Pkg \\
    baseline & Baseline of the badge & \texttt{0mm} & \\
    color & Background color of the right part of the badge & \showColor{default-blue} & \checkmark \\
    colorLeft & Text color of the right part of the badge & \showColor{white} & \checkmark \\
    colorRight & Text color of the right part of the badge & \showColor{white} & \checkmark \\
    labelColor & Background color of the left part of the badge & \showColor{default-left} & \checkmark \\
    logo & Logo displayed before the text & & \\
    logoColor & Color of the logo & \showColor{white} & \checkmark \\
    link & Clickable link & & \\
    showlink & Draw the \textsf{hyperref} border around the badge & \texttt{false} & \\
    style & Style of the badge. Possible values are: \texttt{flat}, \texttt{flat-square}, \texttt{plastic}, \texttt{for-the-badge}, \texttt{social} & \texttt{flat} & \checkmark \\
    \end{tblr}
\end{center}

If the last column is ticked, then the option can be specified as default option for the whole library, that can be passed when loading it, for instance
\mintinline[breaklines]{tex}{\usepackage[labelColor=red]{tikz-shields}}.

Note that anything can be used as a logo, even a text or a personalized command that you would have created (in an example described in Section \ref{sec:special-badges}
\mintinline[breaklines]{tex}{\def\checkmark{\tikz\fill[scale=0.4] (0,.35) -- (.25,0) -- (1,.7) -- (.25,.15) -- cycle;}}).

The Section \ref{sec:examples} presents some examples of badges that can be generated, while Section \ref{sec:special-badges} show some preset badges.





%%%%%%%%%%%%%%%%%%%%%%%%%%%%%%%%%%%%%%%%%%%%%%%%%%%%%%%%%%%%%%%%%%%%%%%%%%%%%%%%%%%%%%%%%%%%%%%%%%%%%%%%%%%%%%%%%%%%%%%%

\section{Examples}
\label{sec:examples}

The color of the right part can be customized, the default color being a rectangle of color \showColor{default-blue}.


\begin{tcblisting}{title={Change color}, lower separated=false}
\drawBadge[color=red]{Any text}{you like}
\drawBadge[labelColor=white, colorLeft=red]{Any text}{you like}
\drawBadge[labelColor=pink, color=black, colorRight=yellow]{Any text}{you like}
\end{tcblisting}


\begin{tcblisting}{title={Customize logo}, lower separated=false}
\drawBadge[logo=\faFilm]{Movie}{5~\faStar}
\drawBadge[logoColor=black, logo=\faHistory, labelColor=yellow]{Any text}{you like}
\def\ctan{\includegraphics[height=1em]{doc/img/ctan-white.pdf}}
\drawBadge[logo=\ctan]{\CTAN}{tikz-shields}
\end{tcblisting}

Custom logos can be employed, as in the previous example where an image is loaded to use the logo%
\footnote{Logo of \CTAN is a courtesy of \href{https://gitlab.com/comprehensive-tex-archive-network/ctan-site/-/blob/master/src/artwork/logo/ctan-book-animation-globe.xcf}{\CTAN}.}.


\begin{tcblisting}{title={Clickable badges}, lower separated=false}
\drawBadge[logo=\faWikipediaW, link=https://www.wikipedia.org]{}{Wikipedia}
\drawBadge[logo=\faWikipediaW, showlink=true, link=https://www.wikipedia.org]{}{Wikipedia}
\end{tcblisting}

Note, for document where the links are hidden by default with \textsf{hyperref} (as the present document), this option won't have any effect.

\vspace{\baselineskip}

The style of the badge can also be chosen:

\begin{tcblisting}{title={Style of badges}, lower separated=false}
\drawBadge[style=flat]{Any text}{you like}
\drawBadge[style=flat-square]{Any text}{you like}
\drawBadge[style=plastic]{Any text}{you like}
\drawBadge[style=for-the-badge]{Any text}{you like}
\drawBadge[style=social]{Any text}{you like}
\end{tcblisting}

As you can see in previous example, the badges are not aligned vertically.
At this point, only \drawBadge[style=social]{social}{badges} are automatically aligned with surrounding text (option \texttt{tcbox raise base} which does not work with \texttt{\textbackslash tcbsidebyside}).

The option \texttt{basline} allows to manually set a vertical shift of the badges, but depend on the text present in the badge, by default it is null.

\begin{tcblisting}{
    title={Baseline
        \footnote{The implementation of the macro \texttt{\textbackslash showbaseline} is taken from \drawBadge[link=https://tex.stackexchange.com/a/432196/243474, logo=\faStackExchange, baseline=1.8mm]{tex.stackexchange.com}{\texttt{432196/243474}}.}},
    lower separated=false}
\showbaseline{Hello, I'm
\drawBadge[baseline=1.8mm]{Arthur}{Pendragon},
and not
\drawBadge[baseline=1mm]{Louis}{XIV}.
}
\end{tcblisting}



%%%%%%%%%%%%%%%%%%%%%%%%%%%%%%%%%%%%%%%%%%%%%%%%%%%%%%%%%%%%%%%%%%%%%%%%%%%%%%%%%%%%%%%%%%%%%%%%%%%%%%%%%%%%%%%%%%%%%%%%

\section{Special badges}
\label{sec:special-badges}


Some presets are available.
For each one of them, the default options can be used to overwrite default values (as logo for instance).

\subsection{Git repository}

\begin{tcblisting}{title={Github badge}, lower separated=false}
\githubBadge{thomas-saigre/tikz-shields}
\githubBadge[logo=\faGit]{thomas-saigre/tikz-shields}
\end{tcblisting}

\begin{tcblisting}{title={Gitlab badge}, lower separated=false}
\gitlabBadge[color=orange]{inkscape/inkscape}
\end{tcblisting}

\begin{tcblisting}{title={Custom forge badge}, lower separated=false}
\forgeBadge[labelColor=black, logo=\faCode]{code.videolan.org/videolan/vlc}
\forgeBadge[labelColor=green, logo=\checkmark]{code.videolan.org/videolan/vlc}
\end{tcblisting}

In all cases, the generated badge is clickable and redirect to the repository provided.
The icon can be customized, as well as the color.

\subsection{License}

\begin{tcblisting}{title={License badges}, lower separated=false}
\licenseBadge[link=https://github.com/thomas-saigre/tikz-shields/blob/main/LICENSE]{GPL-3.0-1}
\licenseBadge[][LICENSE]{cc0}
\licenseBadge[link=]{WTFPL}
\end{tcblisting}

By default, a clickable badge directing to \texttt{https://creativecommons.org/\{name\_of\_the\_license\}} is done.
Note that this webpage may not exist, but the link can either be manually changed, or an empty link can be provided.

The second optional argument allows to personalize the left text, \texttt{license} by default.

